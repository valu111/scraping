const puppeteer = require("puppeteer");
const fs = require("fs");

(async () => {
  const navegador = await puppeteer.launch({
    headless: false,
    slowMo: 80,
    defaultViewport: null,
    args: ["--start-maximized"],
  });

  const pagina = await navegador.newPage();
  const datosEmpleos = [];
  const totalPaginas = 4;

  try {
    for (let numeroPagina = 1; numeroPagina <= totalPaginas; numeroPagina++) {
      const url = `https://hireline.io/mx/empleos?k=contador&l=&page=${numeroPagina}`;
      console.log(`🔎 Explorando página ${numeroPagina}`);

      await pagina.goto(url, { waitUntil: "networkidle2", timeout: 0 });

      await pagina.waitForSelector("ul.jobs-search-results__list li", { timeout: 30000 });

      const empleosPagina = await pagina.evaluate(() => {
        return Array.from(document.querySelectorAll("ul.jobs-search-results__list li")).map((elemento) => {
          const enlace = elemento.querySelector("a");

          const imagen = enlace?.querySelector("div.enterprise-logo img")?.src || "No encontrada";
          const titulo = enlace?.querySelector("p.vacancy-title")?.innerText.trim() || "Sin título";
          const subtitulo = enlace?.querySelector("p.vacancy-subtitle")?.innerText.trim() || "Sin subtítulo";
          const texto = enlace?.querySelector("p.vacancy-summary")?.innerText.trim() || "Sin texto";
          
          // Ubicación y estado de empleo
          const footerItems = elemento.querySelectorAll("div.footer-item");
          let ubicacion = "No encontrada";
          let estadoEmpleo = "No encontrado";

          footerItems.forEach((item) => {
            const icono = item.querySelector("i");
            const textoItem = item.querySelector("p")?.innerText.trim();
            if (icono?.classList.contains("fa-map-marker-alt")) {
              ubicacion = textoItem || ubicacion;
            }
            if (icono?.classList.contains("fa-clock")) {
              estadoEmpleo = textoItem || estadoEmpleo;
            }
          });

          return {
            imagen,
            titulo,
            subtitulo,
            texto,
            ubicacion,
            estadoEmpleo,
            enlace: enlace?.href || ""
          };
        });
      });

      datosEmpleos.push(...empleosPagina);

      console.log(`✅ Página ${numeroPagina} procesada. Empleos encontrados: ${empleosPagina.length}`);
    }



  } catch (error) {
    console.error("❌ Error durante el scraping:", error);
  } finally {
    await navegador.close();
  }
})();




completo de scrapeado:


const puppeteer = require("puppeteer");

(async () => {
  const navegador = await puppeteer.launch({
    headless: false,
    slowMo: 80,
    defaultViewport: null,
    args: ["--start-maximized"],
  });

  const pagina = await navegador.newPage();
  const datosEmpleos = [];
  const totalPaginas = 4;

  try {
    for (let numeroPagina = 1; numeroPagina <= totalPaginas; numeroPagina++) {
      const url = `https://hireline.io/mx/empleos?k=contador&l=&page=${numeroPagina}`;
      console.log(`🔎 Explorando página ${numeroPagina}`);

      await pagina.goto(url, { waitUntil: "networkidle2", timeout: 0 });

      await pagina.waitForSelector("a.hl-vacancy-card.vacancy-container", { timeout: 30000 });

      const empleosPagina = await pagina.evaluate(() => {
        return Array.from(document.querySelectorAll("a.hl-vacancy-card.vacancy-container")).map((elemento) => {
          const imagen = elemento.querySelector("div.enterprise-logo img")?.src || "No encontrada";
          const titulo = elemento.querySelector("p.vacancy-title")?.innerText.trim() || "Sin título";
          const subtitulo = elemento.querySelector("p.vacancy-subtitle")?.innerText.trim() || "Sin subtítulo";
          const texto = elemento.querySelector("p.vacancy-summary")?.innerText.trim() || "Sin texto";

          const footerItems = elemento.querySelectorAll("div.footer-item");
          let ubicacion = "No encontrada";
          let estadoEmpleo = "No encontrado";

          footerItems.forEach((item) => {
            const icono = item.querySelector("i");
            const textoItem = item.querySelector("p")?.innerText.trim();
            if (icono?.classList.contains("fa-map-marker-alt")) {
              ubicacion = textoItem || ubicacion;
            }
            if (icono?.classList.contains("fa-clock")) {
              estadoEmpleo = textoItem || estadoEmpleo;
            }
          });

          return {
            imagen,
            titulo,
            subtitulo,
            texto,
            ubicacion,
            estadoEmpleo,
            enlace: elemento.href || ""
          };
        });
      });

      datosEmpleos.push(...empleosPagina);

      console.log(`✅ Página ${numeroPagina} procesada. Empleos encontrados: ${empleosPagina.length}`);
    }

    // Mostrar resultados como objetos JSON formateados
    datosEmpleos.forEach((empleo) => {
      console.log(JSON.stringify(empleo, null, 2));
    });

  } catch (error) {
    console.error("❌ Error durante el scraping:", error);
  } finally {
    await navegador.close();
  }
})();




fase2

<!DOCTYPE html>
<html lang="es">
<head>
  <meta charset="UTF-8">
  <title>Jugadores</title>
  <!-- Bootstrap CSS desde CDN -->
  <link href="https://cdn.jsdelivr.net/npm/bootstrap@5.3.3/dist/css/bootstrap.min.css" rel="stylesheet">
  <!-- Tu archivo de estilos personalizados -->
  <link rel="stylesheet" href="css/styles.css">
</head>
<body>

  <h1>Lista de jugadores</h1>
  <div id="cards"></div>

  <script>
    fetch('datos.json')
    .then(res => res.json())
    .then(jugadores => {
      const container = document.getElementById("cards");
      jugadores.forEach(j => {
        container.innerHTML += `
          <div class="card">
            <h2>#${j.numeroLista} </h2>
            <h3>${j.nombre}</h3>
            <p>Edad: ${j.edad}</p>
            <p>Puesto: ${j.puesto}</p>
            <p>Fecha: ${j.fecha}</p>
            <p>Valor: ${j.valorMercado}</p>
          </div>
        `;
      });
    });

  </script>

</body>
</html>


potaxie: 
<!DOCTYPE html>
<html lang="es">
<head>
  <meta charset="UTF-8">
  <title>Vacantes</title>
  <link rel="stylesheet" href="stylos.css">
</head>
<body>

  <h1>Vacantes disponibles</h1>
  <div class="contenedor">
    <input type="text" id="busqueda" placeholder="Buscar un Empleo">
  </div>
  
  <div id="cards"></div>

  <script>
    let jugadores = []; 
    const container = document.getElementById("cards");
    const inputBusqueda = document.getElementById("busqueda");

    // Mostrar tarjetas
    function mostrarTarjetas(lista) {
      container.innerHTML = "";
      if (lista.length === 0) {
        container.innerHTML = `
          <div class="card">
            <img src="potaxie.jpg" alt="Sin resultados">
            <h2>No se encontraron resultados</h2>
            <h1>👁️👄👁️</h1>
            <p>Intenta con otros términos de búsqueda .</p>
          </div>
        `;
        return;
      }
      lista.forEach(j => {
        container.innerHTML += `
          <div class="card">
            <img src="${j.imagen}" alt="Imagen del empleo">
            <h1>Título: ${j.titulo}</h1>
            <h3>Subtítulo: ${j.subtitulo}</h3>
            <p>Texto: ${j.texto}</p>
            <p>Ubicación: ${j.ubicacion}</p>
            <p>Estado del empleo: ${j.estadoEmpleo}</p>
            <a href="${j.enlace}" target="_blank">Ver más</a>
          </div>
        `;
      });
    }

    // Cargar datos
    fetch('empleos.json')
      .then(res => res.json())
      .then(data => {
        jugadores = data;
        container.innerHTML = "";
      });

    // Búsqueda
    inputBusqueda.addEventListener("input", () => {
      const textoBusqueda = inputBusqueda.value.trim().toLowerCase();
      if (textoBusqueda === "") {
        container.innerHTML = "";
        return;
      }
      const filtrados = jugadores.filter(j => {
        return (
          j.titulo.toLowerCase().includes(textoBusqueda) ||
          j.subtitulo.toLowerCase().includes(textoBusqueda) ||
          j.texto.toLowerCase().includes(textoBusqueda) ||
          j.ubicacion.toLowerCase().includes(textoBusqueda) ||
          j.estadoEmpleo.toLowerCase().includes(textoBusqueda)
        );
      });
      mostrarTarjetas(filtrados);
    });
  </script>

</body>
</html>






